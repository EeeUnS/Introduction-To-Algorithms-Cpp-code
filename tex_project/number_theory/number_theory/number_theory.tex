\section{기초 정수론}
%https://www.overleaf.com/learn/latex/Aligning%20equations%20with%20amsmath

\begin{justbox}
    \begin{theorem}
        $d,m,n,$이 어떤 정수일 때, 다음이 성립한다.
        \begin{enumerate}
            \item $d$가 $m$과 $n$의 공약수일때, $m+n$도 $d$의 배수이다.
            \item $d$가 $m$과 $n$의 공약수일때, $m-n$도 $d$의 배수이다.     
        \end{enumerate}
    \end{theorem}
\end{justbox}
\begin{proof}
    이에 대한 증명은 간단합니다. 
    $m=dq_1 , n = dq_2( q_1,q_2 \subset Z)$라 하자.
$m+n = d(q_1 + q_2) , m-n=d(q_1-q_2)$ \\
\end{proof}

이 장을 이해하기위해서 약속 몇가지를 정의하겠습니다.
\begin{itemize}
    \item $d$가 $n$의 약수(인수)일 때 $d\: |\: n$으로 표시합니다. 
    \item $m$과 $n$의 최대공약수는 $\gcd(m,n)$이라고 합니다.
    \item $r$이 $a$를 $b$로 나눈 나머지라면  $r=a\bmod b$입니다.
\end{itemize}
이를써서 위 명제를 다시 적으면 $d\mid n , d\mid m  \longrightarrow d \mid (m+n), d\mid (m-n)$        


\begin{justbox}
    \begin{theorem}
        $a,b,z$를 양의 정수라 하면, 다음이 성립한다.
        \[ ab\bmod z= [(a\bmod z)(b \bmod z)]\bmod z \]
    \end{theorem}
\end{justbox}

\begin{proof}

    $w = ab\bmod z$라 하자.
다음이 성립하는 $q_1$이 존재한다.

\[ab=q_1z+w \Longleftrightarrow w=ab-q_1 z \]

마찬가지로 $x =a \bmod z, y=b\bmod z$라 하면, 다음을 만족시키는 $q_2$와 $q_3$, $q$가 존재한다.
$$a=q_2 z + x , b=q_3 z + y$$
\begin{align*}
    w & = ab-q_1 z = (q_2z+x)(q_3z+y)-q_1z\\
    & =(q_2q_3z+q_2y+q_3x-q_1)z+xy\\
    & =qz+xy
\end{align*}
    

여기서 $q=q_2q_3z+q_2y+q_3x-q_1$이므로 
    \[xy=-qz+w\]
즉 $w$는 $xy$를 $z$로 나눌 때의 나머지이다. 그러므로 $w=xy \bmod z$가 되고 이는 다음과 같이 나타낼 수 있다.
     \[ab\bmod z= [(a\bmod z)(b \bmod z)]\bmod z\]    
\end{proof}

이는 큰수를 인수분해해서 작은값으로 나눠서 큰수를 다루는 부담을 덜어주지만 지수승에 대해서도 응용이 가능하다.
이를 이용해서 $a^{29}\bmod z$를 계산하는 절차를 예시로 들어보겠다. $a^{29}$는 다음과 같은 순서로 계산한다.
   \[ a , a^{5}=a \cdot a^4, a^{13}=a^{5}\cdot a^{8}, a^{29}=a^{13}\cdot a^{16} \]
$a^{29} \bmod z$는 다음과 같은 순서로 계산한다.
    
\[a \bmod z , a^{5}\bmod z, a^{13}\bmod z, a^{29}\bmod z\]

\begin{align*}
    a^2 \bmod z & = [(a\bmod z)(a\bmod z)]\bmod z \\
    a^4 \bmod z & = [(a^2\bmod z)(a^2\bmod z)]\bmod z \\
    a^8 \bmod z & = [(a^4\bmod z)(a^4\bmod z)]\bmod z \\
    a^{16} \bmod z & = [(a^8 \bmod z)(a^8\bmod z)]\bmod z \\
    a^5 \bmod z & = [(a \bmod z)(a^4\bmod z)]\bmod z \\
    a^{13} \bmod z & = [(a^5\bmod z)(a^8\bmod z)]\bmod z \\ 
    a^{29} \bmod z & = [(a^{13}\bmod z)(a^{16}\bmod z)]\bmod z
\end{align*}



\section{유클리드 호제법(Euclidean algorithm)}
\begin{justbox}
    \begin{theorem}
        $a$가 음이 아닌 정수이고, $b$가 양의 정수이며 $r=a \bmod b$이면 다음이 성립한다.
\[ \gcd(a,b) = \gcd(b,r) \]  
    \end{theorem}
\end{justbox}
수식이 익숙하지않은 분을 위해 풀어서 설명하자면,
$a$가 음이 아닌 정수이고, $b$가 양의 정수이며, $r$이 $a$를 $b$로 나눈 나머지라면 $a$와 $b$의 최대공약수는 $b$와 $r$의 최대공약수와 같다.
\begin{proof}
$a=bq +r (0 \le r\: <\: b , q$ 는 어떤 정수)인데, $c$를 $a$와  $b$의 공약수라 하면, $c$는 $bq$의 약수인 것은 자명하다.
$a$또한 $c$의 약수이므로 $c$는 $a-bq\:(=r)$의 약수이다. 
따라서 $c$는 $b$와 $r$의 공약수입니다. 반대로 $c'$가 $b$와 $r$의 공약수이면, $c'$는 $bq+r(=a)$의 약수가 되고 따라서 $a$와 $b$의 공약수가 된다다. 
따라서 $a$와 $b$의 공약수 집합이 $b$와 $r$의 공약수 집합과 같으므로 $\gcd(a,b) = \gcd(b,r)$이 성립한다.
\end{proof}

\vspace{1\baselineskip}

유클리드 알고리즘의 의의는 나머지 연산만을 이용해서 뺑뺑이 돌리면 어떻게 됬든지 간에 최대공약수를 기계적으로 구할수있다는 것에 있다. 
$\gcd(a,b) = \gcd(b,r)$에서 b,r을 새로운 a,b로서 값을 넣어서 연속적으로 계산을 하면 언젠가 b가 0이 되는 순간이 오는데,
이때 a가 처음 a,b의 최대 공약수가 되는것이다.


\begin{theorem}
    $\alpha > \beta$일때, 다음이 성립한다.
    
    $\gcd(\alpha ,\beta) = \gcd(\alpha-\beta , \beta)$  
\end{theorem}

\begin{proof}
    $\alpha$, $\beta$의 최대 공약수를  $x$라 하자.
    $\alpha = x \cdot a , \beta = x \cdot b$ ($a,b$는 $a>$b이며 서로소인 두 정수)이며, $\alpha -\beta = x(a-b)$이다 $a-b$는 $b$와 서로소이며 두 값의 최대공약수는 여전히 $x$이다.
\end{proof}




\begin{corollary}
    $f(n) = 1+10+\cdots +10^n$이라 하자.
    
    $ \gcd(f(x) , f(y)) = f(\gcd(x,y)) $임을 보여라.
\end{corollary}
    %...증명

\begin{proof}
    $x > y$라 하자.

    \begin{align*}
        f(x) - f(y) &= 10^x + 10^{x-1} + \cdots + 10^{y+1}\\
        &= 10^y(10^{x-y} + \cdots + 1)\\
        &= f(x-y) \cdot 10^y\\
        \gcd(f(x),f(y)) &= \gcd(f(x)-f(y),f(y)) \\
            &= \gcd(f(x-y) \cdot 10^{y},f(y))
    \end{align*}
    

이때 $10^{y}$와 $f(y)$는 항상 서로소이므로 $\gcd(f(x-y), f(y))$가 성립한다. %...증명

따라서 유클리드 호제법을 전개했을때, $\gcd(f(x), f(y)) = \gcd(f(\gcd(x,y)),0)$이 되고 
이는$f(\gcd(x,y))$과 같다.
\end{proof}


\section{확장된 유클리드 알고리즘(Extended Euclidean algorithm)} 

확장된 유클리드 알고리즘은 다음의 방정식에대해서 $s$와 $t$를 효율적으로 구하는 방법에대한 것이다. 
\begin{justbox}
$a$와 $b$가 음이 아니고 동시에 0이 아닌 정수라 하면 다음을 만족시키는 정수 $s$와 $t$가 존재한다.
\[\gcd(a,b) = s\cdot a + t\cdot b\protect\footnote{선형 디오판토스 방정식이라고도 한다.}\]
\end{justbox}

\subsection{베주의 항등식}

\begin{theorem}
    $ax + by =\gcd(x, y)$인 $a$, $b$가 존재한다.
\end{theorem}
\begin{proof}
    집합 $S = \left\{ m | m =ax+by> , x\in \mathbf{Z} , y \in  \right\}$를 생각해보면 ,이 집합 $S$는 $S \subset \mathbf{Z}$ ,  $S \subset \varnothing$ ( x, y를 원소로 가짐을 알 수 있다.) 이다. 또한, 자연수의 정렬성으로부터 최소가 되는 원소 $d$가 존재한다.

$\alpha \in S \Rrightarrow \alpha = qd+r (0 \le r < d $)라 하자.


만약 $d \nmid \alpha$ 일때, $r > 0$,
$ r = \alpha - qd , (\alpha , d \in S)$ 
$\alpha = a_{1}x+b_{1}y , d=a_{2}x+b_{2}y$라 하면. $r=(a_{1} - a_{2} q)x + (b_{1}-qb_{2})y \in S $
$0 < r < d$인 $r$에 대해 $d$가 최소라는 가정이 모순이다. 

$\therefore r = 0 , d \mid \alpha (\forall \alpha \in S)$,
$ d \mid x, d \mid y \cdots$ $d$는  $x$ , $y$의 공약수,
$\gcd(x, y)=k $라 할때, $d = akx^{''}+bky^{''}=k(ax^{''}+by^{''})$
$k \mid d$에서 $ k = d$
\end{proof}

\subsection{활용}
이미 증명되어있는 유클리드 알고리즘의 흐름을 통해서 예시로 이해 해보자.\\
$a=273$  , $b=110$으로 하는 $\gcd(273,110)$을 구해봅시다.
\begin{center}
    $r= 273\bmod  110 = 53 \cdot\cdots \mathit{1}$
\end{center}
$a=110 , b=53$으로 지정
\begin{center}
    $r= 110\:\bmod \: 53 = 4\cdot\cdots \mathit{2}$
\end{center}
$a=53 , b=4$로 지정
\begin{center}
    $r= 53\:\bmod \: 4 = 1 \cdot\cdots \mathit{3}$
\end{center}
$a=4 , b=1$로 지정
\begin{center}
    $r= 4\bmod  1 = 0\cdot\cdots \mathit{4}$
\end{center}
$r=0$이므로 $\gcd(273,110)$은 최대공약수로 1을 가진다.
여기서 $\mathit{4}$ 식으로 되돌아가면 이는 다음과 같이 쓸 수 있다.
\begin{center}
    $1=53 - 4\cdot13$
\end{center}
계속 역순으로 뒤집어 올라가자 $\mathit{3}$
\begin{center}
    $4=110 - 53\cdot2$
\end{center}
이를 처음의 식에 대입하면
\begin{center}
    $1=53 - (110 - 53\cdot2)\cdot13 =27\cdot53-13\cdot110 $
\end{center}
$\mathit{2}$
\begin{center}
    $53=273 - 110\cdot2$
\end{center}
이 식을 다시 대입하면
\begin{center}
    $1=27\cdot53-13\cdot110=27\cdot(273 - 110\cdot2)-13\cdot110=27\cdot273-67\cdot 110$
\end{center}
따라서 $s=27, t=-67$로서 성립하는 값을 찾았다.




\section{나머지 연산에서 곱셈에 대한 역원 (modular multiplicative inverse)
\protect\footnote{역원: a와 연산자에 대해 연산결과가 항등원($=1$)이 되는 유일한 원소 b를 a의 역원이라한다.}} 

\begin{dfn}[Inverse]
    $\gcd(n,\phi)=1$인 두 정수 $n>0, \phi>1$가 있다고 하자.\footnote{한 마디로 n과 $\phi$는 서로소이다.}
$n\cdot s\bmod  \phi =1 $을 만족시키는 $s$를 $n\bmod  \phi$의 역원(inverse) 이라고 한다.\par
\end{dfn}

$\gcd(n,\phi)=1$임을 이용해, 확장된 유클리드 알고리즘을 이용하여 $s'\cdot n + t \cdot \phi = 1$이되는 $s'$과 $t'$을 구할수있다. 
$n\cdot s'= -t'\phi+1$이 되고 $\phi>1$이므로 1이 나머지가 된다.
$n\cdot s'\bmod  \phi =1$에서 $s= s'\bmod  \phi$라 하면 $0 \le s <\phi$가 되며 또한 $s  \ne 0$이다.\\
위 식을 변형하면, $s'=q\cdot \phi +s$ 가되며 이를 만족하는 정수 q가 존재한다. \\
따라서 
\begin{center}
    $n\cdot s=ns'-\phi nq=-t'\phi +1 -\phi nq=\phi(-t'-nq)+1 $
\end{center}
따라서 $n\cdot s\bmod  \phi =1 $이 된다.




%%%%%%%%%%%%%%%%%%%%%%%%%%%%%%%%%%%%%%%%%%%%%%%%%%%%%%%%%%%%%%%%%%%%%%%%%%%%%%%%%%%%%%%%%%%%%%%%%%%%%%%%%%%%%%%%%%%%%%%%%%%%%%%%%%%%%%%%%%%%%%%%%%%%%%%%%%%%%%%%%%%%%%%%




\section{오일러의 $\phi$함수(Euler’s phi (totient) function)}
\begin{dfn}[phi function]
    양의 정수 $n$에 대해서 

    $\phi (n)$ : 1부터 n까지의 양의 정수 중에 n과 서로소인 것의 개수를 나타내는 함수.    
\end{dfn}

$\phi (n)$은 다음의 성질이 있다.

\begin{theorem}
    
    \begin{itemize}
        \item{소수 $p$에 대해서  $\phi (p)=p-1$}
        
        \item{ m, n이 서로소인 양의 정수일 때, 다음이 성립한다. \[\phi (mn)=\phi (m)\phi (n)\]}
        
        \item 소수 $p$와 양의 정수 $\alpha$에 대해 다음이 성립한다.
        $$\phi(p^\alpha) = p^\alpha \left ( 1 - \frac{1}{p} \right )$$
        \end{itemize}
\end{theorem}

\begin{proof}

첫번째 성질은 어찌보면 당연하다 $p$는 소수이니 자기 자신을 제외한 모든 수와 서로소이다 (여기서 1도 세야한다.)


두번째 성질은 두수의 곱 $mn$은 각각 $m$에대해서 나눠지는 수가 $n$개이고 n에 대해서 나눠지는 수가 $m$개 이며 $mn$으로 나눠지는 수가 한 개이므로 
$mn -\dfrac{mn}{m}-\dfrac{mn}{n}+\dfrac{mn}{mn} =mn -m -n +1=(m-1)(n-1)=\phi (m)\phi(n)$가 된다.

세번째 성질 $p^\alpha$보다 같거나 작은 $p$의 배수가 되는것은 다음이 있다.
$$p , 2p , 3p , ... , p^{\alpha-1}p$$
따라서 총 $p^{\alpha-1}$개가 있고 
$ \phi(p^\alpha) =p^{\alpha} -  p^{\alpha-1} = p^\alpha \left ( 1 - \frac{1}{p} \right )$

\end{proof}

\begin{corollary}

    \emph{If $m_1, m_2, \ldots, m_k$ are $k$ positive
    integers which are prime each to each, then}
    \begin{equation*}
    \phi(m_1 m_2 \ldots m_k) = \phi(m_1) \phi(m_2) \ldots \phi(m_k).
    \end{equation*}
\end{corollary}


\emph{If $m = p_1^{\alpha_1} p_2^{\alpha_2} \ldots p_n^{\alpha_n}$
where $p_1, p_2, \ldots, p_n$ are different primes and $\alpha_1,
\alpha_2, \ldots, \alpha_n$ are positive integers, then}
\begin{equation*}
\phi(m) = m \left ( 1-\frac{1}{p_1} \right )
            \left ( 1-\frac{1}{p_2} \right )
            \ldots
            \left ( 1-\frac{1}{p_n} \right ).
\end{equation*}

For,
\begin{align*}
\phi(m) &= \phi(p_1^{\alpha_1}) \phi(p_2^{\alpha_2}) \ldots
             \phi(p_n^{\alpha_n}) \\
        &= p_1^{\alpha_1} \left ( 1-\frac{1}{p_1} \right )
             p_2^{\alpha_2} \left ( 1-\frac{1}{p_2} \right )
             \ldots
             p_n^{\alpha_n} \left ( 1-\frac{1}{p_n} \right ) \\
        &= m \left ( 1-\frac{1}{p_1} \right )
             \left ( 1-\frac{1}{p_2} \right )
             \ldots
             \left ( 1-\frac{1}{p_n} \right ).
\end{align*}



\section{오일러 정리(Euler's theorem)\protect\footnote{페르마의 소정리는 오일러 정리에서의 특수한 경우이다.}}

\begin{justbox}
    \begin{theorem}
        임의의 정수 a와 n이 서로소일 때, 다음이 성립한다.
        \[a^{\phi(n)} \bmod n = 1\]
    \end{theorem}
\end{justbox}

\begin{proof}
    
정수 n에 대해서 1부터 n까지의 양의 정수 중에 n과 서로소인 것의 집합을 생각해보자.
그러면 이는 집합
\begin{center}
    $A = \{ r_1 ,r_2,r_3, \cdots ,r_{\phi(n)}\}$\footnote{이러한 집합을 기약잉여계 $\mathbb Z_n^*$라고 부른다. 또한 집합 A의 원소의 갯수는 $\phi(n)$dl다.}
\end{center}
으로 나타낼 수 있다. 이 집합은 A라하고 이 각 원소에 n과 서로소인 a를 곱한 집합을 B집합이라 하자.
\[B = \{ ar_1 ,ar_2,ar_3, \cdots ,ar_{\phi(n)}\} \]
확실한건 $B$에 있는 모든 원소는 $n$과 서로소인 것이다. 
그럼 $B$집합의 각 원소를 $\bmod n$에 대해 계산한 것을 생각해보자.
 이는 각 원소의 나머지가 a를 곱하기전 값과 같은지는 모르지만 $\phi(n)$개에 대해서 각각 일대일대응이 가능 한다는것을 알수있다.
  \footnote{실제 증명은 귀류법을 통해서 증명할수있다.$ar_i  \equiv ar_j \bmod n $ 인 $1 \le i < j \le \phi(n)$ 이 존재한다고 가정해보자.}
따라서 $A$의 모든 원소를 곱한 값에 $\bmod n$을 한것과 $B$의 모든 원소를 곱한 값에 $\bmod n$을 한 값은 같다.
\[ar_1 \cdot ar_2 \cdot ar_3 \cdots ar_{\phi_{n}} \equiv r_1 \cdot r_2 \cdot r_3 \cdots r_{\phi_{n}} \pmod n\]
\[a^{\phi(n)}\bmod n= 1\]
\end{proof}


\begin{corollary}
    페르마의 소정리 :  소수 $p$에 대해 다음이 성립한다.
    $$ a^{p-1} \equiv 1 \pmod{p}$$
\end{corollary}

\begin{proof}
    $\phi(p)  = p-1 $이므로 오일러 정리에의해 성립합을 알 수 있다.
\end{proof}


\begin{equation*}
    a^{p-1} \equiv 1 \bmod p.
\end{equation*}
    
%%%%%%%%%%%%%%%%%%%%%%%%%%%%%%%%%%%%%%%%%%%%%%%%%%%%%%%%%%%%%%%%%%%%%%%%%%%%%%%%%%%%%%%%%%%%%%%%%%%%%%%%%%%%%%%%%%%%%%%%%%%%%%%%%%%%%%%%%%%%%%%%%%%%%%%%%%%%%%%%%%%%%%%%
