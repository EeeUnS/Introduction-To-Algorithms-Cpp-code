\documentclass{oblivoir}
    \usepackage{ikps,ansform}
    \usepackage{lipsum}
  
    \newcounter{problem}[section]
    \newenvironment{problem}{\noindent\refstepcounter{problem}\textbf{\large\theproblem.} }{}
    
\begin{document}
\par
\title{중국인의 나머지 정리 (Chinese Remainder Theorem)}
\author{ EUnS }
\maketitle
$x \equiv a_1 \pmod{m_1}$
$x \equiv a_2 \pmod{m_2}$
$x \equiv a_n \pmod{m_n} ( \forall i,j \gcd(m_{i} ,m_{j}) = 1$\footnote{서로소 아이디얼 (pairwise coprime)}$)$
일때, $x$가 $\Zeta / m_{1}m_{2} \cdots m_{n}$에서 유일하게 존재
\chapter{존재성}
$m = m_{1}m_{2} \cdots m_{n}, n_{k} = \frac{m}{n_k}$을 생각하면 
$t_{k}m_{k}+s_{k}n_{k} = 1$인 정수 $s_{k} ,t_{k}$가 존재한다 $( \because \gcd( m_{k}, n_{k}) = 1 )$ \footnote{베주 항등식}
$s_{k}n_{k} \equiv 1 \pmod{m_k}$
$x=a_{1}n_{1}s_{1} + \cdot + a_{n}n_{n}s_{n} = \sum_{k=1}^n a_{k}n_{k}s_{k}$
$j \ne k \longrightarrow m_{k} \mid n_{j} \longrightarrow x \equiv a_{k}n_{k}s_{k} \equiv a_{k} \pmod(m_{k})$
\chapter{유일성}
귀류법을 사용한다.
서로다른 $x$, $y$가 $\mod m$에서 합동식의 해라 하자.
$x \equiv y\equiv a_1 \pmod{m_1}$
$x \equiv y\equiv a_2 \pmod{m_2}$
$x \equiv y\equiv a_n \pmod{m_n}$
$x - y \equiv 0 \pmod{m_k} ( 1\le k \le n$인 정수 $k )$
$\operatorname{lcm}(m_1, m_2, \cdots, m_n) \mid (x-y) \longrightarrow m_{1} m_{2} \cdots m_{n}( = m ) \mid (x-y) (\forall i,j \gcd(m_{i} ,m_{j}) = 1)$
$\therefore x \equiv y \pmod{\operatorname{lcm}(m_1, m_2, \cdots, m_n)}$
이는 모순이다.
\end{document}