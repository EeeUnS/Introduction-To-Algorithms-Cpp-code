\usepackage{framed}

다항식과 fft

\section{개론}

n차 다항식 $A(x)$를 예시로 들라고하면 대부분 이렇게 대답할것이다.

$A(x) = a_0x^0 + a_1x^1 + a_2x^2 + \cdots + a_nx^n = \sigma_{k=0}^{n} a_kx^k$

이는 컴퓨터상에서 크기가 $0~n$인 벡터(배열)에 $(a_0,a_2, ... , a_n )$으로 나타내도 $A(x)$를 확정지을 수 잇다. 이러한 표현 방식을 \textbf{계수 표현()}이라고 한다.
우리 일반적으로 사용하는 다항식을 나타내는 방식이라고 생각하면되는데. 
이렇게 나타낸 다항식을 각각 곱하는 경우를 생각해보자.

일반적으로 우리는 종이에서 다음 다항식을 곱할때 이와같이 풀것이다.

$C(x) = A(x)B(x) = (a_0x^0 + a_1x^1 + a_2x^2 + \cdots + a_nx^n)(b_0x^0 + b_1x^1 + b_2x^2 + \cdots + b_nx^n)$ 

2차나 3차의 경우는 한번에 쭉 풀수도있겠지만

n이 큰 경우일때는 어쩔수없이 $A(x)$ 항하나에 $B(x)$를 곱해서 쭉 전개해서 풀것이다.

이를 직접 컴퓨터에서 계산한다고 생각해보자.
벡터 각요소에 벡터 각요소를 각각 모두 곱하기 때문에 시간복잡도는 $O(n^2)$이 될것이다. 이것이 우리가 일반적으로 생각하는 다항식 곱의 시간복잡도이다

이를 $O(n \log n)$으로 줄여 볼 것이다.

\section{점값 표현}

중,고등학교를 나오면서 다음과 같은 문제를 푼적이 있을거라고 생각합니다.

\begin{framed}
    \begin{itemize}
        \item 2차원 좌표상에서 $(1,0)$과 $(6,8)$을 지나는 직선의 방정식을 구하시오.
        \item 2차원 좌표상에서 $(1,0)$, $(6,8)$,$(-4,8)$ 을 지나는 다항함수를 구하시오.
    \end{itemize}
    
\end{framed}
간단한 연립방정식처럼 생각하면 최고차항에 따라서 어떤 다항식을 지나는 $n+1$개 이상의 점의 좌표를 알고있으면 그 다항식을 특정할수있다.

컴퓨터상에서도 그대로 나타낼수있습니다.
$\left\{\left\{x_1, y_1\right\},\left\{x_2, y_2\right\}, ... ,\left\{x_n, y_n\right\}\right\}$

이를 \textbf{점값 표현}이라고 합니다.




\section{계수표현 점값표현 치환}



\section{DFT와 FFT}




\section{DFT^{-1} 역변환}
