\documentclass{oblivoir}

%%%%%그림 그리기 어려워서 동결
\usepackage{graphicx}


\definecolor{mGreen}{rgb}{0,0.6,0}
\definecolor{mGray}{rgb}{0.5,0.5,0.5}
\definecolor{mPurple}{rgb}{0.58,0,0.82}
\definecolor{backgroundColour}{rgb}{0.95,0.95,0.92}
%https://tex.stackexchange.com/questions/348651/c-code-to-add-in-the-document
\lstdefinestyle{CStyle}{
    backgroundcolor=\color{backgroundColour},   
    commentstyle=\color{mGreen},
    keywordstyle=\color{magenta},
    numberstyle=\tiny\color{mGray},
    stringstyle=\color{mPurple},
    basicstyle=\footnotesize,
    breakatwhitespace=false,         
    breaklines=true,                 
    captionpos=b,                    
    keepspaces=true,                 
    numbers=left,                    
    numbersep=5pt,                  
    showspaces=false,                
    showstringspaces=false,
    showtabs=false,                  
    tabsize=2,
    language=C
}


\title{RED-BLACK-TREE DELETETION}
\author{EUnS}

\begin{document}

레드블랙트리의 기본은 알고있다고 생각하고 진행하는데, 한번만 더 레드블랙트릐의 규칙을 한번 짚어 봅시다.

\begin{itemize}
    \item 노드 색깔은 RED or BLACK
    \item ROOT노드의 색깔은 BALCK
    \item 리프 노드(NIL)색깔은 BLACK
    \item RED 색깔 노드의 자식 노드의 색깔은 BLACK이다.
    \item ROOT노드에서 리프 노드 까지 도달하는 BLACK노드의 갯수는 같다.
\end{itemize}

의사코드에선 다음의 약속을 합니다
노드 u와 트리 T에 대해
\begin{itemize}
    \item u.p : u의 부모 자식
    \item u.left : u의 왼쪽 자식
    \item u.right : u의 오른쪽 자식
    \item T.root : 트리의 루트노드
    \item T.NIL : 트리의 리프 노드
    \item u.color : u의 색깔
\end{itemize}


또한 추가적인 프로시저가 사용됩니다.


\begin{lstlisting}[style = CStyle]
RB-TRANSPLANT(T,u,v);
    if u.p = T.nil
        T.root = v
    elseif u == u.p.left
        u.p,left = v
    else u.p,right = v
    v.p = u.p
\end{lstlisting}

u를 v로 단순히 대체만 합니다.


\begin{lstlisting}[style = CStyle]
LEFT-ROTATE(T, x)
    y = x.right // set y
    x.right = y.left // turn y’s left subtree into x’s right subtree
    if y.left != T.nil
        y.left.p = x
    y.p = x.p // link x’s parent to y
    if x.p == T.nil
        T.root = y
    elseif x == x.p.left
        x.p.left = y
    else x.p.right = y
    y.left = x // put x on y’s left
    x.p = y
\end{lstlisting}
다음은 ROTATE 프로시저의 작동 방식 입니다.
%%%%%%%%%%%%%%%%%%%%%%

\begin{figure}[h!]
    \centering
    \includegraphics[scale=0.3]{{quick1.png}}
\end{figure}


RIGHT_ROTATE는 좌우 반대로 구현하면 됩니다


TREE_MINIMUM(u) : u노드 자식 중 가장 최솟값을 반환합니다.



\begin{lstlisting}[style = CStyle]
RB_DELETE(T,z)
    y = z
    y_original_color = y.color
    if z.left == T.nil
        x = z.right
        RB_TRANSPLANT(T, z, z.right)
    elseif z.right == T.nil
        x = z.left
        RB_TRANSPLANT(T, z, z.left)
    else y = TREE_MINIMUM(z.right)
        y_original_color = y.color
        x = y.right
        if y.p == z
            x.p = y
        else RB_TRANSPLANT(T, y , y.right)
            y.right = z.right
            y.right.p = y
        RB_TRANSPLANT(T, z, y)
        y.left = z.left
        y.left.p = y
        y.color = z.color
    if y_original_color = BLACK
        RB_DELETE_FIXUP(T, x)
\end{lstlisting}

Case 1:





\begin{thebibliography}{}
    \bibitem{reference1}
    Thomas H. Cormen, Charles E. Leiserson, Ronald L. Rivest, and Clifford Stein. Introduction to Algorithms, Second Edition. MIT Press and McGraw-Hill, 2001. ISBN 0-262-03293-7.
\end{thebibliography}

\end{document}