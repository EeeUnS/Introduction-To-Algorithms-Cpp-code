
    1. Prove that $\lambda(n) \mid \phi(n)$.
    
    $n = p_1^{e_1} \cdots p_r^{e_r}$

    $ \phi(n) = \phi(p_1^{e_1})* \ldots*\phi(p_r^{e_r})$

    $\text{lcm}(\phi(p_1^{e_1}, \ldots, \phi(p_r^{e_r})) | (\phi(p_1^{e_1})* \ldots*\phi(p_r^{e_r}))$
    
    $\lambda(n) \mid \phi(n)$

    2. Prove that Carmichael numbers must be both “square-free” (not divisible by the square of any prime) 

    let Carmichael number $n = p^\alpha m( \alpha \ge 2 ,  p \nmid m )$
    $a^{n-1} \equiv 1 \pmod{n} (\gcd(a,n) = 1)$
    
    set $a = p+1 $ then $(p+1)^{n} \equiv p+1 \pmod{n}$
    
    and $gcd(p^2,a) = 1$
    
    $(p+1)^{n} \equiv (p+1)^{p^2 p^{\alpha-2} \equiv p+1 \pmod{p^2}$
    
    but $\gcd(p^2,a) = 1$,  $ a \equiv 1 \pmod{p^2}$
    
    $p+1 \equiv 1 \pmod{p^2}$ This is impossible
    
    
    3. the product of at least three primes. 
    
    Assume that $n=pq$, with $p<q$ two distinct primes, is a Carmichael number. 
    Then we have 
    $q≡1 \pmod{q−1} )\rightarrow n \equiv pq \equiv p \pmod{q−1}  \rightarrow n−1 \equiv p−1 \pmod{q−1}$
    Here $0 < p−1 < q−1$ , so $n−1$ is not divisible by $q−1$.