

\section{RSA시스템의 이해}

\subsection{RSA 공개 키 암호 시스템 
\protect\footnote{로널드 라이베스트(Ron Rivest), 아디 샤미르(Adi Shamir), 레너드 애들먼(Leonard Adleman)이 세명의 이름 앞글자를 따서 지었다.}
(RSA public-key cryptosystem)}

이 알고리즘은 보안 기법중 하나로 가장 흔한 예시로서는 공인인증서가 있다.
    \[A \longrightarrow B\]
$A$가 $B$에게 숫자를 하나 보낸다고 생각 해보자. $A$에게는 공개키가 필요하며 $B$에게는 개인키가 있어야한다. 공개키는 누가 가져도 상관없는 키이며 개인키는 절대로 노출되어서는 안되는 키이다.\\
$A$는 $B$에게 $a$를 보낼때 공개키를 이용하여 $a$를 $c$로 암호화 하여 보내며 $B$는 $c$를 공개키와 개인키를 이용하여 $a$로 복호화하여 읽는 방식이다.



\subsection{공개키, 암호키 생성}
두 개의 소수 $p,q$를 선택하여 $n=pq$를 계산한다.
\footnote{그 후 $p ,q$는 버린다. 가지고 있어봤자 개인키가 뚫리는 취약점이 될수가있다.}
 그 후 $\phi =(p-1)(q-1)$을 계산하고 $\gcd(n,\phi)=1$인 정수 $e$을 선택한다. 그후 $n$과 $e$를 공개한다.
  $ed\bmod \phi =1$이고 $0<d<\phi$를 만족시키는 $d$를 생성하여 $d$를 개인키로 사용한다.
\footnote{d는 위에서 언급한 나머지 연산에서 곱셈에 대한 역원을 구하는 방법으로 효율적으로 구할수있다.}\\


\subsection{단계}
$A$가 $B$에게 정수 $a(0\le a\le z-1)$를 보내기 위해서 $A$는 $c=a^e \bmod n$ 를 계산하여 $c$를 보낸다.
\footnote{c를 효율적으로 구하는 방법 또한 위에서 다루었다.}
$B$는 $c^d \bmod n$를 계산하면 이 값이 $a$이다.\\





\subsection{복호화 과정}

$ \phi(n) = \phi$

$ ed\bmod \phi =1 \Longleftrightarrow ed = b\phi(n)+1$($b$는 어떤 상수)

\begin{align*}
  c^d \bmod n &=(a^e \bmod n)^d \bmod n \\ 
  &=(a^e)^d \bmod n = a^{ed} \bmod n \\
  &=  a^{b\phi(n)+1}\bmod n  \\
  &= (a^{\phi(n)} \bmod n)^{b} a \bmod n =a  
\end{align*}

\footnote{오일러정리 사용}

\subsection{이게 과연 안전한가?}


이를 구하기위한 해결방법은 결과적으로 소인수분해와 직결되는데 그냥 n을 p와 q로소인수 분해해버리면 끝난다. 그러나 소인수분해를 다항시간내에하는 알고리즘은 개발되지 않았다.
